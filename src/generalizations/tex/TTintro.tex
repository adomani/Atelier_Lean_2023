\documentclass{beamer}
\usepackage{xcolor}

\usetheme{Boadilla}
\usefonttheme{serif}

\usepackage{amsmath}
\usepackage{amssymb}
\usepackage{tikz}
\usepackage{minted}

\usepackage{hyperref}
\hypersetup{
    colorlinks=true,       % false: boxed links; true: colored links
    linkcolor=blue,        % color of internal links
    citecolor=magenta,     % color of links to bibliography
    filecolor=green,       % color of file links
    urlcolor=purple        % color of external links
}

\usepackage{newunicodechar}
\newunicodechar{ℕ}{\ensuremath{\mathbb{N}}}
\newunicodechar{ℤ}{\ensuremath{\mathbb{Z}}}
\newunicodechar{Ñ}{\ensuremath{\mathcal{N}}}
\newunicodechar{ᵐ}{\ensuremath{^m}}
\newunicodechar{ᶜ}{\ensuremath{^c}}
\newunicodechar{ᶠ}{\ensuremath{^f}}

\title{Introduction to Type Theory}
\author{Damiano Testa}
\institute[]{University of Warwick}
\date[\href{http://www.rnta.eu/7MSRNTA/lean.html}{Atelier Lean 2023}]{\\
\vspace{50pt}
RNTA mini symposium
\href{http://www.rnta.eu/7MSRNTA/lean.html}{Atelier Lean 2023}\\
\vspace{20pt}
May 3rd, 2023
}


\begin{document}

\frame{\titlepage}

\begin{frame}[fragile]
{Introduction to Type Theory}

This talk is an extended digression on Type Theory.
\bigskip

As is usually the case, foundations of mathematics have a marginal impact on ``real-world'' mathematics.
\bigskip

This is true also when using Lean
\bigskip

$\ldots$ most of the times!
\end{frame}

\begin{frame}[fragile]{Set Theory}

Most mathematicians learn that Set Theory is the foundation of mathematics.

This normally comes with
\bigskip

\begin{itemize}
\item
  a more or less ``primitive'' concept of a {\color{violet}\verb`set`};
\item
  the {\color{violet}\verb`belongs-to`} relation $\in$ among sets;
\item
  an empty set;
\item
  several rules for constructing new sets from old ones.
\end{itemize}
\bigskip

Mathematics is then built on top of these foundations (often assuming the existence of an infinite set).
\end{frame}

\begin{frame}[fragile]{Everything is a set}

Whenever we introduce a new concept, we practice Set Theory by making sure that ``it is a set'':

\begin{itemize}
\item
  the natural numbers are a set,
\item
  ordered pairs are a set,
\item
  the real numbers are a set,
\item
  functions are a set,
\item
  sequences are functions (and hence are a set),
\item
  $\ldots$
\end{itemize}
\bigskip

After all, everything is a set. Everything. EVERYTHING.
\bigskip
\end{frame}

\begin{frame}[fragile]{Set Theory -- really?}

I imagine that most mathematicians {\emph{could}} explain how to encode their favourite mathematical concept using only the rules for forming sets out of old sets, starting from the empty set.
\bigskip

I also imagine that most mathematicians would rather {\emph{not}} do that.
\bigskip

Descending inside all the nested sets of sets of sets until we reach the empty set is probably detrimental to developing an intuition about modular forms, or schemes.
\end{frame}

\begin{frame}[fragile]

We may ``undo'' one nesting or two, but, after that, we probably stop and think about {\color{violet}\verb`structured`} sets.
\bigskip

For instance, if $f \colon A \longrightarrow B$ is a function, we may think of it as a way of converting an element of the set $A$ to an element of the set $B$.

$\to$ No undoing: a {\color{violet}\verb`structured`} function.
\bigskip

Sometimes, it can be useful to think of the ``graph of $f$'' (which, in Set Theory {\color{violet}\verb`is`} the function).

$\to$ 1 undoing: {\color{violet}\verb`structured`} ordered pairs.
\bigskip

How often have you then used the Kuratowski's encoding of ordered pairs to {\color{violet}\verb`really`} understand $f$?

$\to$ 2 undoings: {\color{violet}\verb`unstructured`} chaos.
\end{frame}

\begin{frame}[fragile]{Types as structured sets}

From the perspective of Set Theory,
\begin{itemize}
\item
  a {\color{violet}\verb`Type`} is like a set,
\item
  its elements are called {\color{violet}\verb`terms`},
\item
  the {\color{violet}\verb`belong-to`} relation is denoted by {\color{violet}\verb`:`}.
\end{itemize}
\bigskip

Thus, if {\color{violet}\verb`t`} is a term of a Type {\color{violet}\verb`T`}, we can write {\color{violet}\verb`t : T`}.
\bigskip

A fundamental axiom is that every term has a unique Type.
\bigskip

Types usually come with rules, called {\color{violet}\verb`constructors`}, for building terms of the given Type.
\bigskip

The constructors endow their Type with some internal {\color{violet}\verb`structure`}.
\bigskip

Let's see the definition of natural numbers in Lean.
\end{frame}

\begin{frame}[fragile]

\begin{minted}[mathescape, numbersep=5pt, frame=lines, framesep=2mm, fontsize=\small]{Lean}
inductive myℕ
  | zero : myℕ
  | succ : myℕ → myℕ
\end{minted}
{\small{\href{https://leanprover-community.github.io/lean-web-editor/#code=inductive%20my%E2%84%95%0A%20%20%7C%20zero%20%3A%20my%E2%84%95%0A%20%20%7C%20succ%20%3A%20my%E2%84%95%20%E2%86%92%20my%E2%84%95%0A%0A%23print%20prefix%20my%E2%84%95%0A}{Click here to open the Lean web editor}.}}
\bigskip

The code above defines a Type {\color{violet}\verb`myℕ`}.
\bigskip

The Type {\color{violet}\verb`myℕ`} contains an element (really, a {\color{violet}\verb`term`}), that we call {\color{violet}\verb`zero`}.
\bigskip

We also postulate the existence of a function {\color{violet}\verb`succ`} from {\color{violet}\verb`myℕ`} to {\color{violet}\verb`myℕ`}.
\bigskip

Lean's Type Theory takes care of making {\color{violet}\verb`myℕ`} ``universal''.
\end{frame}

\begin{frame}[fragile]

 is a Type, containing

 most mathematicians who are not logicians, is this
\end{frame}
\end{document}
