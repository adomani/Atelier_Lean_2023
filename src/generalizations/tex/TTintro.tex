\documentclass{beamer}
\usepackage{xcolor}

\usepackage[skip=22pt plus1pt, indent=0pt]{parskip}

\usetheme{Boadilla}
\usefonttheme{serif}

\usepackage{amsmath}
\usepackage{amssymb}
\usepackage{tikz}
\usepackage{minted}

\usepackage{hyperref}
\hypersetup{
    colorlinks=true,       % false: boxed links; true: colored links
    linkcolor=blue,        % color of internal links
    citecolor=magenta,     % color of links to bibliography
    filecolor=green,       % color of file links
    urlcolor=purple        % color of external links
}

\usepackage{newunicodechar}
\newunicodechar{ℕ}{\ensuremath{\mathbb{N}}}
\newunicodechar{ℤ}{\ensuremath{\mathbb{Z}}}
\newunicodechar{Ñ}{\ensuremath{\mathcal{N}}}
\newunicodechar{ᵐ}{\ensuremath{^m}}
\newunicodechar{ᶜ}{\ensuremath{^c}}
\newunicodechar{ᶠ}{\ensuremath{^f}}

\title{Introduction to Type Theory}
\author{Damiano Testa}
\institute[]{University of Warwick}
\date[\href{http://www.rnta.eu/7MSRNTA/lean.html}{Atelier Lean 2023}]{\\
\vspace{50pt}
RNTA mini symposium
\href{http://www.rnta.eu/7MSRNTA/lean.html}{Atelier Lean 2023}\\
\vspace{20pt}
May 3rd, 2023
}


\begin{document}

\frame{\titlepage}

\begin{frame}[fragile]
{Introduction to Type Theory}

This talk is an extended digression on Type Theory.

As is usually the case, foundations of mathematics have a marginal impact on ``real-world'' mathematics.

This is true also when using Lean

$\ldots$ most of the times!
\end{frame}

\begin{frame}[fragile]{Set Theory}

Set Theory is a common choice of foundation for mathematics.

This normally comes with

\vspace{-15pt}
\begin{itemize}
\setlength\itemsep{-18pt}
\item
  a more or less ``primitive'' concept of a {\color{violet}\verb`set`}
\item
  the {\color{violet}\verb`belongs-to`} relation $\in$ among sets
\item
  several rules for constructing new sets from old ones
\item
  an empty set.
\end{itemize}

Mathematics is then built on top of these foundations.
\end{frame}

\begin{frame}[fragile]{Everything is a set}

We practice Set Theory by ensuring that ``everything is a set'':

\vspace{-15pt}
\begin{itemize}
\setlength\itemsep{-18pt}
\item
  the natural numbers are a set
\item
  ordered pairs are a set
\item
  the real numbers are a set
\item
  functions are a set
\item
  sequences are functions (and hence are a set)
\item
  $\ldots$
\end{itemize}

Everything is a set. Everything. EVERYTHING.
\end{frame}

\begin{frame}[fragile]{Set Theory -- really?}

Most mathematicians {\color{violet}\verb`can`} explain how to encode their favourite mathematical concept using only the basic axioms of set theory.

Most mathematicians would probably {\color{violet}\verb`not want`} to do that.

{\color{violet}\verb`Descending`} inside all the nested sets of sets of sets until we reach the empty set is probably even {\color{violet}\verb`detrimental`} to developing an {\color{violet}\verb`intuition`} about modular forms, schemes, or any {\color{violet}\verb`advanced`} mathematical concept.
\end{frame}

\begin{frame}[fragile]

We may {\color{violet}\verb`undo`} one nesting or two, but, after that, we probably stop and think about {\color{violet}\verb`structured`} sets.

If $f \colon A \longrightarrow B$ is a {\color{violet}\verb`function`}, we may think of it as a rule to {\color{violet}\verb`convert`} elements of $A$ to elements of $B$. \\
$\to$ No undoing: a {\color{violet}\verb`structured`} function.

Sometimes, it can be useful to think of the {\color{violet}\verb`graph`} of $f$. \\
In Set Theory this {\color{violet}\verb`is`} the function. \\
$\to$ 1 undoing: {\color{violet}\verb`structured`} ordered pairs.

Have you ever used Kuratowski's encoding of ordered pairs to {\color{violet}\verb`really`} understand $f$? \\
$\to$ 2 undoings: {\color{violet}\verb`unstructured`} chaos.
\end{frame}

\begin{frame}[fragile]{Types as structured sets}

From the perspective of Set Theory,
\vspace{-15pt}
\begin{itemize}
\setlength\itemsep{-18pt}
\item
  a {\color{violet}\verb`Type`} is like a set
\item
  its elements are called {\color{violet}\verb`terms`}
\item
  the {\color{violet}\verb`belong-to`} relation is denoted by {\color{violet}\verb`:`}.
\end{itemize}

Thus, if {\color{violet}\verb`t`} is a term of a Type {\color{violet}\verb`T`}, we can write {\color{violet}\verb`t : T`}.

A fundamental axiom is that every term has a unique Type.

Types usually come with rules, called {\color{violet}\verb`constructors`}, for building terms of the given Type.

The constructors endow their Type with some internal {\color{violet}\verb`structure`}.

Let's see the definition of natural numbers in Lean.
\end{frame}

\begin{frame}[fragile]

\begin{minted}[mathescape, numbersep=5pt, frame=lines, framesep=2mm, fontsize=\small]{Lean}
inductive myℕ
  | zero : myℕ
  | succ : myℕ → myℕ
\end{minted}
{\small{\href{https://leanprover-community.github.io/lean-web-editor/#code=inductive%20my%E2%84%95%0A%20%20%7C%20zero%20%3A%20my%E2%84%95%0A%20%20%7C%20succ%20%3A%20my%E2%84%95%20%E2%86%92%20my%E2%84%95%0A%0A%23print%20prefix%20my%E2%84%95%0A}{Click here to open the Lean web editor}.}}
\bigskip

The code above defines a Type {\color{violet}\verb`myℕ`}.
\bigskip

The Type {\color{violet}\verb`myℕ`} contains an element (really, a {\color{violet}\verb`term`}), that we call {\color{violet}\verb`zero`}.
\bigskip

We also postulate the existence of a function {\color{violet}\verb`succ`} from {\color{violet}\verb`myℕ`} to {\color{violet}\verb`myℕ`}.
\bigskip

Lean's Type Theory takes care of making {\color{violet}\verb`myℕ`} ``universal''.
\bigskip

For instance, Lean auto-generates the induction principle.
\end{frame}

\begin{frame}[fragile]

In Lean's Type Theory, there is an inbuilt axiom:

\vspace{-15pt}
\begin{itemize}
\setlength\itemsep{-18pt}
\item
  {\emph{every}} term has a {\emph{unique}} Type.
\end{itemize}
\bigskip

For instance, the Type {\color{violet}\verb`myℕ`} above contains the term {\color{violet}\verb`zero`} (really, the term is {\color{violet}\verb`myℕ.zero`}).
\bigskip

Imagine that we eventually we define {\color{violet}\verb`myℤ.zero`}.
\bigskip

The two terms {\color{violet}\verb`myℕ.zero : myℕ`} and {\color{violet}\verb`myℤ.zero : myℤ`} are {\emph{different}}.
\end{frame}

\begin{frame}[fragile]

We can make Lean aware of the unique homomorphism {\color{violet}\verb`myℕ → myℤ`}.
\bigskip

However, we can pretend that {\color{violet}\verb`myℕ.zero`} and {\color{violet}\verb`myℤ.zero`} are ``the same'', only as long as some tactic takes care of converting between the two.
\bigskip

Notice also that in Set Theory, the usual definitions of
$$
  0 \in \mathbb{N} \qquad {\textrm{and}} \qquad 0 \in \mathbb{Z}
$$
yield {\emph{different}} elements.
\bigskip

Even the containment $\mathbb{N} \subset \mathbb{Z}$ is false.
\bigskip

Type Theory simply makes us more aware of these (usually inconsequential) inconsistencies.
\end{frame}

\begin{frame}[fragile]{Why many proof checkers use Type Theory?}

Using a proof checker, ultimately means writing a computer program to verify a mathematical reasoning.

In Set Theory, {\textbf{many}} syntactically correct statements are garbage.

For instance, deciding whether the relations
$$
  \mathbb{N} \in \pi
  \qquad {\textrm{or}} \qquad
  \mathbb{Q}_{\le 0} \subset e
  \qquad {\textrm{or}} \qquad
  \sqrt{2} ^ 2 = \emptyset
$$
hold is ``meaningful''.

In Type Theory, none of the above {\color{violet}\verb`Type-checks`}.
\end{frame}

\begin{frame}[fragile]

$$
  \mathbb{N} \in \pi
  \qquad {\textrm{or}} \qquad
  \mathbb{Q}_{\le 0} \subset e
  \qquad {\textrm{or}} \qquad
  \sqrt{2} ^ 2 = \emptyset
$$
\bigskip

In the background, Lean is constantly Type-checking every assertion.
\bigskip

This means that it can alert us to the fact that we are writing ``non-sense'' {\emph{before}} a proof-checker based on Set Theory would.
\bigskip

You can think of {\color{violet}\verb`Type-checking`} as {\color{violet}\verb`dimensional-analysis`} in physics:

\smallskip
\[
  \left[ \;
  \parbox{0.8\textwidth}{\centering\small
  if you compute the speed of your car to be $70$Kg, you are sure that you've made a mistake!
  }
  \; \right]
\]
\bigskip
\end{frame}

\begin{frame}[fragile]{Type Theory vs Set Theory}

To me, the most insightful consequence of formalizing mathematics in Type Theory came from focusing on the separation:

 mathematical concept
\end{frame}

\begin{frame}[fragile]

Using Set Theory or Type Theory as foundations, has little bearing on the theorems that we can prove.

Sometimes, it may lead us to prefer one approach over another.

In practice, the main difference of Type Theory using Lean vs ``''
\end{frame}

\begin{frame}[fragile]
\end{frame}
\end{document}
