\documentclass{beamer}
\usepackage{xcolor}

\usepackage[skip=15pt plus1pt, indent=0pt]{parskip}

\usetheme{Boadilla}
\usefonttheme{serif}

\usepackage{amsmath}
\usepackage{amssymb}
\usepackage{tikz}
\usepackage{minted}

\usepackage{hyperref}
\hypersetup{
    colorlinks=true,       % false: boxed links; true: colored links
    linkcolor=blue,        % color of internal links
    citecolor=magenta,     % color of links to bibliography
    filecolor=green,       % color of file links
    urlcolor=purple        % color of external links
}

\usepackage{newunicodechar}
\newunicodechar{ℕ}{\ensuremath{\mathbb{N}}}
\newunicodechar{ℝ}{\ensuremath{\mathbb{R}}}
\newunicodechar{ℤ}{\ensuremath{\mathbb{Z}}}

\newunicodechar{Ñ}{\ensuremath{\mathcal{N}}}
\newunicodechar{ᵐ}{\ensuremath{^m}}
\newunicodechar{ᶜ}{\ensuremath{^c}}
\newunicodechar{ᶠ}{\ensuremath{^f}}

\title{Introduction to Type Theory}
\author{Damiano Testa}
\institute[]{University of Warwick}
\date[\href{http://www.rnta.eu/7MSRNTA/lean.html}{Atelier Lean 2023}]{\\
\vspace{50pt}
RNTA mini symposium
\href{http://www.rnta.eu/7MSRNTA/lean.html}{Atelier Lean 2023}\\
\vspace{20pt}
May 3rd, 2023
}


\begin{document}

\frame{\titlepage}

\begin{frame}[fragile]
{Introduction to Type Theory}

This talk is an extended digression on {\color{violet}\verb`Type Theory`}.

As is usually the case, {\color{violet}\verb`foundations`} of mathematics have a {\color{violet}\verb`marginal impact`} on ``real-world'' mathematics.

This is true also when using {\color{violet}\verb`Lean`}

$\ldots$ most of the times!
\end{frame}

\begin{frame}[fragile]{Set Theory}

{\color{violet}\verb`Set Theory`} is a common choice of {\color{violet}\verb`foundation`} for mathematics.

This normally comes with

\vspace{-13pt}
\begin{itemize}
\setlength\itemsep{-12pt}
\item
  a more or less ``primitive'' concept of a {\color{violet}\verb`set`};
\item
  the {\color{violet}\verb`belongs-to`} relation $\in$ among sets;
\item
  several rules for constructing new sets from old ones;
\item
  an empty set.
\end{itemize}

{\color{violet}\verb`Mathematics`} is then built on top of these {\color{violet}\verb`foundations`}.
\end{frame}

\begin{frame}[fragile]{Everything is a set}

We practice Set Theory by ensuring that {\color{violet}\verb`everything is a set`}:

\vspace{-13pt}
\begin{itemize}
\setlength\itemsep{-12pt}
\item
  the {\color{violet}\verb`natural numbers`} are a {\color{violet}\verb`set`},
\item
  {\color{violet}\verb`ordered pairs`} are a {\color{violet}\verb`set`},
\item
  the {\color{violet}\verb`real numbers`} are a {\color{violet}\verb`set`},
\item
  {\color{violet}\verb`functions`} are a {\color{violet}\verb`set`},
\item
  {\color{violet}\verb`sequences`} are functions (and hence are a {\color{violet}\verb`set`}),
\item
  $\ldots$
\end{itemize}

{\color{violet}\verb`Everything`} is a set. {\color{violet}\verb`Everything`}. {\color{violet}\verb`EVERYTHING`}.
\end{frame}

\begin{frame}[fragile]{Set Theory -- really?}

Most mathematicians {\color{violet}\verb`can`} explain how to {\color{violet}\verb`encode`} their favourite mathematical concept using only the {\color{violet}\verb`basic`} axioms of {\color{violet}\verb`set theory`}.

Most mathematicians would probably {\color{violet}\verb`not want`} to do that.

{\color{violet}\verb`Descending`} inside all the nested sets of sets of sets until we reach the empty set is probably even {\color{violet}\verb`detrimental`} to developing an {\color{violet}\verb`intuition`}.

Imagine doing it for {\color{violet}\verb`modular forms`}, {\color{violet}\verb`schemes`}, or any {\color{violet}\verb`advanced`} mathematical concept.
\end{frame}

\begin{frame}[fragile]

We may {\color{violet}\verb`undo`} one nesting or two, but, after that, we probably stop and think about {\color{violet}\verb`structured`} sets.

If $f \colon A \longrightarrow B$ is a {\color{violet}\verb`function`}, we may think of it as a rule to {\color{violet}\verb`convert`} elements of $A$ to elements of $B$.
\vspace{-13pt}
\begin{itemize}
\setlength\itemsep{-12pt}
\item
  No undoing: a {\color{violet}\verb`structured`} function.
\end{itemize}

Sometimes, it can be useful to think of the {\color{violet}\verb`graph`} of $f$.
\\
In Set Theory this {\color{violet}\verb`is`} the function.
\vspace{-13pt}
\begin{itemize}
\setlength\itemsep{-12pt}
\item
  1 undoing: {\color{violet}\verb`structured`} ordered pairs.
\end{itemize}

Have you ever used Kuratowski's encoding of ordered pairs to {\color{violet}\verb`really`} understand $f$?
\vspace{-13pt}
\begin{itemize}
\setlength\itemsep{-12pt}
\item
  2 undoings: {\color{violet}\verb`unstructured`} chaos.
\end{itemize}
\end{frame}

\begin{frame}[fragile]{Types as structured sets}

\vspace{10pt}
\vspace{-13pt}
\begin{itemize}
\setlength\itemsep{-12pt}
\item
  A {\color{violet}\verb`Type`} is like a set,
\item
  its ``elements'' are called {\color{violet}\verb`terms`},
\item
  the {\color{violet}\verb`belong-to`} relation is denoted by {\color{violet}\verb`:`} (a colon).
\end{itemize}

Thus, {\color{violet}\verb`t : T`} means that {\color{violet}\verb`t`} is a term of a Type {\color{violet}\verb`T`}.

A fundamental axiom is that {\color{violet}\verb`every term`} has a {\color{violet}\verb`unique`} Type.

Each Type come with rules, called {\color{violet}\verb`constructors`}, to build its terms.
\\
The constructors endow their Type with some internal {\color{violet}\verb`structure`}.

Let's see the definition of the {\color{violet}\verb`natural numbers`} in Lean.
\end{frame}

\begin{frame}[fragile]

\begin{minted}[mathescape, numbersep=5pt, frame=lines, framesep=2mm, fontsize=\small]{Lean}
inductive myℕ
  | zero : myℕ
  | succ : myℕ → myℕ
\end{minted}
\vspace{-17pt}
{\small{\href{https://leanprover-community.github.io/lean-web-editor/#code=inductive%20my%E2%84%95%0A%20%20%7C%20zero%20%3A%20my%E2%84%95%0A%20%20%7C%20succ%20%3A%20my%E2%84%95%20%E2%86%92%20my%E2%84%95%0A%0A%23print%20prefix%20my%E2%84%95%0A}{Click here to open the Lean web editor}.}}

The code above defines a Type {\color{violet}\verb`myℕ`}.

The Type {\color{violet}\verb`myℕ`} contains an element (really, a {\color{violet}\verb`term`}), that we call {\color{violet}\verb`zero`}.

We also postulate the existence of a function {\color{violet}\verb`succ`} from {\color{violet}\verb`myℕ`} to {\color{violet}\verb`myℕ`}.

Lean's Type Theory takes care of making {\color{violet}\verb`myℕ`} ``universal''.

For instance, Lean {\color{violet}\verb`auto-generates`} the {\color{violet}\verb`induction`} principle.
\end{frame}

\begin{frame}[fragile]{Every {\color{violet}\texttt{term}} has a unique {\color{violet}\texttt{Type}}}

\begin{minted}[mathescape, numbersep=5pt, frame=lines, framesep=2mm, fontsize=\small]{Lean}
inductive myℕ
  | zero : myℕ
  | succ : myℕ → myℕ
\end{minted}

In Lean's Type Theory, there is an inbuilt {\color{violet}\verb`axiom`}:

\vspace{-13pt}
\begin{itemize}
\setlength\itemsep{-12pt}
\item
  {\color{violet}\verb`every`} term has a {\color{violet}\verb`unique`} Type.
\end{itemize}

The Type {\color{violet}\verb`myℕ`} contains the term {\color{violet}\verb`zero`} (really, the term is {\color{violet}\verb`myℕ.zero`}).

Imagine that eventually we define {\color{violet}\verb`myℤ.zero`}.

The two terms {\color{violet}\verb`myℕ.zero : myℕ`} and {\color{violet}\verb`myℤ.zero : myℤ`} are {\emph{different}}.
\end{frame}

\begin{frame}[fragile]

We {\color{violet}\verb`can`} make Lean aware of the unique homomorphism {\color{violet}\verb`myℕ → myℤ`}.

However, we {\color{violet}\verb`can't`} pretend that {\color{violet}\verb`myℕ.zero`} and {\color{violet}\verb`myℤ.zero`} are ``the same'', unless some {\color{violet}\verb`tactic`} takes care of the {\color{violet}\verb`conversion`}.

Also in Set Theory, the usual definitions of
$$
  0 \in \mathbb{N} \qquad {\textrm{and}} \qquad 0 \in \mathbb{Z}
$$
yield {\color{violet}\verb`different`} elements.

{\emph{A fortiori}}, the {\color{violet}\verb`containment`} $\mathbb{N} \subset \mathbb{Z}$ is {\color{violet}\verb`false`}.

{\color{violet}\verb`Type Theory`} simply makes us more {\color{violet}\verb`aware`} of these (usually inconsequential) inconsistencies.
\end{frame}

\begin{frame}[fragile]

We highlight another consequence of ``Every term has a unique Type''.

In Set Theory, the {\color{violet}\verb`real number`} $1 \in \mathbb{R}$ is also {\color{violet}\verb`contained`} in the set of {\color{violet}\verb`non-negative real numbers`}: $1 \in \mathbb{R}_{\ge 0}$.

In Type Theory, the term {\color{violet}\verb`1`} belongs to {\color{violet}\verb`at most one`} of $\mathbb{R}$ and $\mathbb{R}_{\ge 0}$.

(Of course, both $\mathbb{R}$ and $\mathbb{R}_{\ge 0}$ have their unit, but they are {\color{violet}\verb`different`}.)

In {\color{violet}\verb`mathlib`}'s formalization, the Type $\mathbb{R}_{\ge 0}$ is a type of pairs
$$
  (x, h) \; : \; \mathbb{R}_{\ge 0}
$$
where {\color{violet}\verb`x : ℝ`} is a real number and {\color{violet}\verb`h`} is a proof of the inequality $0 \le x$.
\end{frame}

\begin{frame}[fragile]{Why many proof checkers use Type Theory?}

Using a {\color{violet}\verb`proof checker`}, ultimately means writing a {\color{violet}\verb`computer program`} to {\color{violet}\verb`verify`} mathematical {\color{violet}\verb`reasoning`}.

In Set Theory, {\textbf{many}} syntactically correct statements are {\color{violet}\verb`garbage`}.

For instance, deciding whether the relations
$$
  \left(
    {\textrm{Norm}} \colon \mathbb{Q}\left( \sqrt{2} \right) \to \mathbb{Q}
  \right) \in \pi,
  \quad {\textrm{or}} \qquad
  \mathbb{Q}_{\le 0} \subset e,
  \quad {\textrm{or}} \qquad
  \sqrt{2} ^ 2 = \emptyset.
$$
hold is ``meaningful''.
Usually, {\color{violet}\verb`no one cares`} about the answers.

In Type Theory, none of the above {\color{violet}\verb`Type-checks`}.
\end{frame}

\begin{frame}[fragile]{Type-checking feedback}

$$
  \left(
    {\textrm{Norm}} \colon \mathbb{Q}\left( \sqrt{2} \right) \to \mathbb{Q}
  \right) \in \pi,
  \quad {\textrm{or}} \qquad
  \mathbb{Q}_{\le 0} \subset e,
  \quad {\textrm{or}} \qquad
  \sqrt{2} ^ 2 = \emptyset.
$$

In the background, {\color{violet}\verb`Lean`} constantly {\color{violet}\verb`Type-checks`} every assertion.

This means that it can {\color{violet}\verb`alert`} us to the fact that we are writing ``non-sense'' {\color{violet}\verb`before`} a proof-checker based on Set Theory would.

You can think of {\color{violet}\verb`Type-checking`} as {\color{violet}\verb`dimensional-analysis`} in physics:
\[
  \left[ \;
  \parbox{0.8\textwidth}{\centering\small
  if you compute the speed of your bike to be $12$Kg, you are sure that you've made a mistake!
  }
  \; \right]
\]

$\ldots$ and Lean will let you know.
\end{frame}

\begin{frame}[fragile]{Implementation details}

Formalizing mathematics made me focus on the separation:

{\centering
\begin{tabular}{|c|c|}
\hline
{\color{violet}\verb`Platonic world`} &  {\color{violet}\verb`Real-world mirror`}  \\
\hline
mathematical {\color{violet}\verb`concept`} & {\color{violet}\verb`realization`} in set theory \\
abstract {\color{violet}\verb`idea`}        & {\color{violet}\verb`implementation detail`} \\
\hline
\end{tabular}

}

{\textbf{Example}}. {\color{violet}\verb`Implementations`} of the polynomial ring $\mathbb{Z}[x]$:
\\
\vspace{-13pt}
\begin{itemize}
\setlength\itemsep{-12pt}
\item
  formal, linear combinations of symbols $\{ x^n \}_{n \in \mathbb{N}}$;
\item
  ``meaningful'', finite $\mathbb{Z}$-linear sums of the power functions $\{ x^n \}_{n \in \mathbb{N}}$;
\item
  Finitely supported functions $\mathbb{Z} \to \mathbb{N}$ with the convolution product;
\item
  A commutative ring with unit representing the forgetful functor
\end{itemize}
$$
  \begin{array}{rcl}
    {\textbf{CommRings}} & \longrightarrow & {\textbf{Sets}} \\
    R & \longmapsto & R
  \end{array}
$$
\end{frame}

\begin{frame}[fragile]{Type Theory vs Set Theory}

The {\color{violet}\verb`distinction`} between Set Theory and Type Theory as foundations for mathematics is an {\color{violet}\verb`implementation detail`}.

Most of the times, it {\color{violet}\verb`does not matter`}.

Indeed, {\color{violet}\verb`foundations`} are almost {\color{violet}\verb`invisible`} (unless you focus on logic).

This applies to {\color{violet}\verb`pen-and-paper`}, as well as {\color{violet}\verb`formalized`} mathematics.

\[
  \left[ \;
  \parbox{0.8\textwidth}{\centering\small
  Lean's version of Type Theory is {\color{violet}\texttt{equiconsistent}} with \\``ZFC + there exist countably many inaccessible cardinals''.
  }
  \; \right]
\]
\end{frame}

\begin{frame}[fragile]{Conclusion}

The {\color{violet}\verb`foundations`} used for formalization, should be {\color{violet}\verb`invisible`}.

They mostly are, in practice, with surprisingly few exceptions.

When these {\color{violet}\verb`exceptions`} arise, they often bring an {\color{violet}\verb`insight`} on some topic that you thought had no {\color{violet}\verb`surprises`} left!
\bigskip
\bigskip

\LARGE\centerline{
{\textbf{Questions}}?
}
\end{frame}
\end{document}
