\documentclass{beamer}
\usepackage{xcolor}

\usepackage[skip=15pt plus1pt, indent=0pt]{parskip}

\usetheme{Boadilla}
\usefonttheme{serif}

\usepackage{amsmath}
\usepackage{amssymb}
\usepackage{tikz}
\usepackage{minted}

\usepackage{hyperref}
\hypersetup{
    colorlinks=true,       % false: boxed links; true: colored links
    linkcolor=blue,        % color of internal links
    citecolor=magenta,     % color of links to bibliography
    filecolor=green,       % color of file links
    urlcolor=purple        % color of external links
}

\usepackage{newunicodechar}
\newunicodechar{ℕ}{\ensuremath{\mathbb{N}}}
\newunicodechar{ℤ}{\ensuremath{\mathbb{Z}}}
\newunicodechar{Ñ}{\ensuremath{\mathcal{N}}}
\newunicodechar{ᵐ}{\ensuremath{^m}}
\newunicodechar{ᶜ}{\ensuremath{^c}}
\newunicodechar{ᶠ}{\ensuremath{^f}}

\title{Automation in Lean}
\author{Damiano Testa}
\institute[]{University of Warwick}
\date[\href{http://www.rnta.eu/7MSRNTA/lean.html}{Atelier Lean 2023}]{\\
\vspace{50pt}
RNTA mini symposium
\href{http://www.rnta.eu/7MSRNTA/lean.html}{Atelier Lean 2023}\\
\vspace{20pt}
May 2nd, 2023
}


\begin{document}

\frame{\titlepage}

\begin{frame}[fragile]
{Automation}

Computers take on repetitive tasks.

In the context of formalization of mathematics, the computer also

\vspace{-13pt}
\begin{itemize}
\setlength\itemsep{-12pt}
\item
  helps producing more complicated arguments, as it separates neatly different parts of the argument;
\item
  informs, ideally, the {\textbf{discovery}} of new mathematical results;
\item
  detects {\emph{very well}} unnecessary hypotheses.
\end{itemize}

\[
  \left[ \;
  \parbox{0.8\textwidth}{\centering\small
  The resulting generality is often only useful to simplify {\emph{formalization}}, rather than {\emph{discovery}} of mathematics.
  }
  \; \right]
\]

Currently, Machine Learning, Artificial Intelligence, Neural Networks and auto-formalizations are not yet really available.

There is lots of interest and steady progress on this front.
\end{frame}

\begin{frame}[fragile]{Tactics}

{\emph{Any}} tactic is a form of automation.

Tactics allow to maintain abstraction:

\vspace{-13pt}
\begin{itemize}
\setlength\itemsep{-12pt}
\item
  we humans talk about mathematical concepts,
\item
  the computer has some representation for these concepts.
\end{itemize}

Tactics bridge this gap.

We do not need to know what the computer's internal representation is: tactics handle the translation.
\end{frame}

\begin{frame}[fragile]

In the previous talks, you have already seen some tactics ({\color{violet}\verb`exact`}, {\color{violet}\verb`intro`}, {\color{violet}\verb`apply`}, {\color{violet}\verb`rw`}, ...).

Now, we talk about {\color{violet}\verb`library_search`} and {\color{violet}\verb`simp`}.

These tactics probably feel closer to an intuitive idea of ``automation'' that you may have.
\end{frame}

\begin{frame}[fragile]{{\color{violet}\texttt{library\_search}}}

{\color{violet}\verb`mathlib`} is a massive repository: it contains
\vspace{-13pt}
\begin{itemize}
\setlength\itemsep{-12pt}
\item
  over 1 million lines of code
\item
  over 60 thousand lemmas.
\end{itemize}

Most of the basic{\footnotemark} lemmas are already available.

{\color{violet}\verb`library_search`} helps you find them!

\footnotetext{``Basic'' may mean {\emph{really}} basic, to a level that you may not even consider them ``lemmas''.}
\end{frame}

\begin{frame}[fragile]

\begin{minted}[mathescape, numbersep=5pt, frame=lines, framesep=2mm, fontsize=\small]{Lean}
import tactic

example {a b c : ℕ} : a ^ (b + c) = a ^ b * a ^ c :=
by library_search

--  Try this: exact pow_add a b c
\end{minted}

\vspace{-17pt}
{\small{\href{https://leanprover-community.github.io/lean-web-editor/#code=import%20tactic%0A%0Aexample%20%7Ba%20b%20c%20%3A%20%E2%84%95%7D%20%3A%20a%20%5E%20%28b%20%2B%20c%29%20%3D%20a%20%5E%20b%20*%20a%20%5E%20c%20%3A%3D%0Aby%20library_search}{Click here to open the Lean web editor}.}}

\bigskip

Besides {\color{violet}\verb`library_search`}, {\color{violet}\verb`mathlib`} has a very helpful \href{https://leanprover-community.github.io/contribute/naming.html}{naming convention} that allows you to ``guess'' names of lemmas.
\end{frame}

\begin{frame}[fragile]{The {\color{violet}\texttt{simp}}-lifier}

As the name suggests, the {\color{violet}\verb`simp`}-lifier tries to simplify a goal.

\begin{minted}[mathescape, numbersep=5pt, frame=lines, framesep=2mm, fontsize=\small]{Lean}
import tactic

example {a b : ℤ} :
  - (-1 * a + 0 * b) = a * (1 + a * 0) :=
begin
  simp,
end
\end{minted}

\vspace{-17pt}
{\small{\href{https://leanprover-community.github.io/lean-web-editor/#code=import%20tactic%0A%0Aexample%20%7Ba%20b%20%3A%20%E2%84%A4%7D%20%3A%0A%20%20-%20%28-1%20*%20a%20%2B%200%20*%20b%29%20%3D%20a%20*%20%281%20%2B%20a%20*%200%29%20%3A%3D%0Abegin%0A%20%20simp%2C%0Aend}{Click here to open the Lean web editor}.}}

{\color{violet}\verb`simp`} automatically used the lemmas

\begin{tabular}{|l|l|l|l|}
\hline
 {\color{violet}\verb`mul_zero`} & {\color{violet}\verb`zero_mul`} & {\color{violet}\verb`one_mul`} & {\color{violet}\verb`mul_one`} \\
{\color{violet}\verb`neg_neg`} & {\color{violet}\verb`neg_mul`} & {\color{violet}\verb`add_zero`} &\\
\hline
\end{tabular}

\end{frame}

\begin{frame}[fragile]{``{\color{violet}\texttt{simp}}-lemmas'': lemmas that {\color{violet}\texttt{simp}} uses}
\vphantom{}
\vspace{-13pt}
\begin{itemize}
\setlength\itemsep{-12pt}
\item
  They assert an equality or an iff.
\item
  The LHS {\textbf{looks more complicated}} than the RHS.
\end{itemize}
\vspace{-20pt}
\begin{minted}[mathescape, numbersep=5pt, frame=lines, framesep=2mm, fontsize=\small]{Lean}
#print mul_zero  -- means:   a * 0 = 0
#print zero_mul  -- means:   0 * a = 0
#print one_mul   -- means:   1 * a = a
#print mul_one   -- means:   a * 1 = a
#print neg_neg   -- means:    - -a = a
#print neg_mul   -- means:  -a * b = -(a * b)
#print add_zero  -- means:   a + 0 = 0
\end{minted}
\vspace{-10pt}
The asymmetry helps Lean to flow along
$$
  {\texttt{hard LHS}} \longrightarrow {\texttt{easy RHS}}.
$$
\[
  \left[ \;
  \parbox{0.8\textwidth}{\centering\small
  Being a ``{\color{violet}\texttt{simp}}-lemma'' is something that {\emph{you}} must communicate to Lean: there is no automated mechanism that makes Lean self-select which lemmas are {\color{violet}\texttt{simp}}-lemmas.
  }
  \; \right]
\]
\end{frame}

\begin{frame}[fragile]{{\color{violet}\texttt{simp}}-normal-form and confluence}

{\color{violet}\verb`simp`}-lifying LHS to RHS leads to questions of {\color{violet}\verb`confluence`}.

Ideally, {\color{violet}\verb`simp`} invariably {\color{violet}\verb`converges`} to an ``optimal'' final shape.

In reality, there are practical and theoretical reasons why this cannot be the case.

Still, {\color{violet}\verb`simp`} is a very useful automation tool.

``Locally'', it achieves normalization efficiently and effectively.
\end{frame}

\begin{frame}[fragile]

Let's switch over to an \href{https://leanprover-community.github.io/lean-web-editor/#url=https%3A%2F%2Fraw.githubusercontent.com%2Fadomani%2FAtelier_Lean_2023%2Fadomani_polys%2Fsrc%2Fgeneralizations%2F1.generalizations.presentationTemplate.lean}{interactive demo}.
\end{frame}
\end{document}
